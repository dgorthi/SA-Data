\documentclass[12pt,usletter,english]{article}
\usepackage[utf8]{inputenc}
\usepackage{graphicx}
\usepackage{amsmath}
\usepackage{amssymb}
\usepackage{listings}
\usepackage{float}
\usepackage{hyperref}
\hypersetup{
  colorlinks=true,
  linkcolor=blue,
  filecolor=magenta,
  urlcolor=cyan,
}

\urlstyle{same}

\begin{document}
\title{Voltage Recorder Setup} \author{Deepthi Gorthi} \maketitle

Gbt-1u (the 1U Server in the box), has five 1G ethernet ports and one
10G (SFP+) ethernet port. The five 1G ethernet can be located, on the
back panel of the server, with respected to the VGA port according to
the table below.

\begin{center}
  \begin{tabular}{| l | c | c | r |}
    \hline
    eno5 & eno3 & eno4 & \\
    \hline
    & eno1 & eno2 & VGA \\
    \hline
  \end{tabular}
\end{center}

The ports have been configured for the following connections:

\paragraph{IP addresses}
\begin{description}
\item[eno1] DHCP. Should be connected to the Netgear Switch (GS748T),
  to any of the private ports (1-32). This will let me access gbt-1u
  through Paper1 which is connected to the
  internet. \href{https://docs.google.com/presentation/d/1pqmiAIBzzXq71BUphQWKdoJBhLlbmE_w2vNg_Hh2KaE/edit#slide=id.p}{Link to diagram on HERA Wiki.}
\item[eno2] static, 10.1.0.1 -- This port can be used to connect to
  gbt-1u with a laptop for debugging network problems.
\item[eno3] static, 10.10.0.100 -- Port to be connected to the
  raspberry pi. I gave the raspberry pi a compatible static IP (10.10.0.233).
\item[enp1s0] 10gb port, 10.10.10.10 -- to be connected to one of the
  SFP+ ports on the SNAP. The SNAP SFP+ ports are currently wired to the
  Arista 40GbE switch. The computer can be connected to the same
  switch to establish a connection to the SNAP. There should be spare
  cables in the container. If not, the cable going to hera-digi can be
  borrowed.
\end{description}

\paragraph {Antennas}

There are 12 antennas, all north polarizations only. They can be
connected in order to the 12 SNAP inputs. The SNAP board inputs from
left to right are given in the following table (read the table
row-wise left to right, there is only one row on the SNAP). There's a pretty picture of this at 
\url{https://casper.berkeley.edu/wiki/images/f/fe/SNAPv1_labled_assembly.pdf}

\begin{center}
\begin{tabular}{|c|c|c|c|}
  \hline
  (Beside SFP+) 1PPS Out   & 1PPS In       & Synth Ref       & Clock In        \\ \hline
  (SMA 5)  ADC Input 3  & (SMA 6) ADCI 2   & (SMA 7) ADCI 1  & (SMA 8) ADCI 0  \\ \hline
  (SMA 9)  ADC Input 7  & (SMA 10) ADCI 6  & (SMA 11) ADCI 5 & (SMA 12) ADCI 4 \\ \hline
  (SMA 13) ADC Input 11 & (SMA 14) ADCI 10 & (SMA 15) ADCI 9 & (SMA 16) ADCI 8 \\ \hline
\end{tabular}
\end{center}

I extracted the roach inputs, corresponding to the antennas I want, on
Oct 1 (1.30 pm PST). If they have changed since then, I can update
this table. I does not matter if the SNAP ADC Inputs are connected in
order or SMA Inputs are connected in order as long as it is noted
down. Nevertheless I've added a prefered order in the last column of
the table below- the SMA input number is counted assuming the one
right next to the SFP+ port is 1.

\begin{center}
\begin{tabular}{|c|c|c|c|}
  \hline
  Antenna Number & Roach Input & Panel Fxin & SNAP SMA \\
  \hline
  24   &   D F7 A2   &  p6 r1 c1  & 8  \\  \hline
  25   &   D F7 A4   &  p6 r1 c2  & 7  \\  \hline
  26   &   D F7 B2   &  p6 r1 c3  & 6  \\  \hline
  27   &   D F7 B4   &  p6 r1 c4  & 5  \\  \hline
  52   &   D F8 C2   &  p5 r5 c5  & 12 \\  \hline
  53   &   D F8 D2   &  p5 r5 c6  & 11 \\  \hline
  54   &   D F7 H2   &  p5 r5 c7  & 10 \\  \hline
  55   &   D F5 D4   &  p4 r1 c3  & 9  \\  \hline
  84   &   D F8 F2   &  p5 r5 c1  & 16 \\  \hline
  85   &   D F8 G2   &  p5 r5 c2  & 15 \\  \hline
  86   &   D F8 B2   &  p5 r5 c3  & 14 \\  \hline
  87   &   D F8 A2   &  p5 r5 c4  & 13 \\  \hline
\end{tabular}
\end{center}

\paragraph {Software}
All the debugging and observation software is on
\url{https://github.com/dgorthi/SA-Data}.

The most useful one is SA-Data/obs/obs\_script.py which programs the
SNAP board and lets you see the output spectra of all the analog
inputs.

In an ipython session type the following commands:
\begin{lstlisting}
  \%run obs_script.py 46 54 76 #chans to select
  plot_chans() #for plotting all the analog inputs
  plotspec(ant=<ant no.>) # To plot one analog input
\end{lstlisting}

\end{document}
